\documentclass[12pt,a4paper,oneside]{article}
\usepackage{amsmath, amsthm, amssymb, graphicx,indentfirst,bm} % Required for inserting images
\newtheorem{theorem}{theory}[section]
\usepackage[bookmarks=true, colorlinks, citecolor=blue, linkcolor=black]{hyperref}

\usepackage{geometry}
\geometry{left=2.54cm,right=2.54cm,top=3.18cm,bottom=3.18cm}

\linespread{1.5}

\pagenumbering{roman}

\setcounter{page}{0}

\title{Practice}
\author{FengJiacheng}
\date{\today}

\begin{document}

\maketitle

\tableofcontents

\section{Introduction}
\subsection{first subsection}
first subsection

\subsection{sec subsection}
second subsection

\subsection{fig}
this is fig.1

this is fig.1

htbp rules for selecting the best place for a picture automatically

\begin{figure}[htbp]
    \centering
    \includegraphics[width=8cm]{12.png}
    \caption{Model 1}
\end{figure}

\subsection{Form}
this is form

this is form
\begin{table}[htbp]
\centering
\caption{form 1}
    \begin{tabular}{|l|l|l|l|l|}
    \hline
    1 &   &   &     \\ \hline
      & 1 &   &     \\ \hline
      &   & 1 &     \\ \hline
      &   &   & 1   \\ \hline
    \end{tabular}
\end{table}

\subsection{List}
this is a list,enumerate is the list with order

this is a list,enumerate is the list with order

\begin{enumerate}
    \item this is the first point;
    \item this is the second point;
    \item this is the third point;
\end{enumerate}

Also some other types.

\begin{enumerate}
    \item[(1)] first; 
    \item[/2/] second;
    \item[[3]] third;
\end{enumerate}

\subsection{theory}
\begin{theorem}[moll]
Some Theories

\subsection{fomula}
$a>0$,$b>0$,so$a+b>0$

$\displaystyle\lim_{n\to\infty}x_n=x$

If $a>0$,$b>0$,so:
\[
a+b>0
\]

up mark and down mark:
\[
A^2 \quad \text{and} \quad A_2
\]

and \quad $\dfrac{a}{b}$

also \quad $c^\frac{a}{b}$

kuohao$(1+a)$

large kuohao $\left(1+\dfrac{1}{n}\right)^n$

highlight \bm{$a>0$}

long kuohao 
\[
f(x)=
\begin{cases}
x,  &  x>0,\\
-x, &  x<0.
\end{cases}
\]

multies rows:
\[
\begin{aligned}
    a & =b+c\\
      & =d+e
\end{aligned}
\]

matrix environments bmatrix and pmatrix:
\[
\begin{pmatrix}
    a & b\\
    c & d
\end{pmatrix}
\]
\end{theorem}



















\end{document}
